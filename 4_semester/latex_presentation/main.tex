\documentclass{beamer}
\usetheme{default}

\usepackage{xcolor}
\usepackage{blindtext}
\usepackage{indentfirst}
\usepackage{lipsum}
\usepackage[utf8]{inputenc}
\usepackage[russian]{babel}

\begin{document}
    \onehalfspacing
    \setlength{\parskip}{\baselineskip}%
    \setlength{\parindent}{0.8pt}%

    \title{Множества. Комбинаторика.}
    \author{Aminjon Shermatov}

    \begin{frame}
        \titlepage
    \end{frame}

    \begin{frame}{Contents}
        \doublespacing
        \tableofcontents
        \singlespacing
    \end{frame}

    \section{Множества}

    \subsection{Определение}
    \begin{frame}{Определение}
        \fontsize{10}{8}\selectfont Буквами \textit{\textbf{N, Z, Q, R, C}} обозначают соответственно множества натуральных, целых, рациональных, действительных и комплексных чисел

        Если \textit{\textbf{x}} -- элемент множества \textit{\textbf{A}}, то пишут $\textbf{\textit{x}} \in \textbf{\textit{A}}$, а если \textit{\textbf{x}} не является элементом множества \textit{\textbf{A}}, то пишут $\textbf{\textit{x}} \notin \textbf{\textit{A}}$

        Если каждый элемент множества \textit{\textbf{A}} является элементом множества \textit{\textbf{B}}, то пишут $\textit{\textbf{A}} \subset \textit{\textbf{B}}$ или $\textit{\textbf{B}} \supset \textit{\textbf{A}}$ и говорят, что множество \textit{\textbf{A}} является подмножеством множества \textit{\textbf{B}}. В этом случае говорят также, что \textit{\textbf{A}} содержится в \textit{\textbf{B}} или что \textit{\textbf{B}} содержит \textit{\textbf{A}}.

        Если $\textit{\textbf{A}} \subset \textit{\textbf{B}}$ и $\textit{\textbf{B}} \subset \textit{\textbf{A}}$, то $\textit{\textbf{A}} = \textit{\textbf{B}}$.

        Для удобства вводится понятие пустого множества (его обозначают $\O$), которое по определению не содержит элементов и содержится в любом множестве.
    \end{frame}

    \subsection{Операции над множествами}
    \begin{frame}{Операции над множествами}
        \fontsize{10}{8}\selectfont Множество, состоящее из всех тех и только тех элементов, которые принадлежат хотя бы одному из множеств \textit{\textbf{A}} и \textit{\textbf{B}}, называется
        объединением множеств \textit{\textbf{A}} и \textit{\textbf{B}} и обозначается $\textbf{\textit{A}} \cup \textbf{\textit{B}}$ или $\textbf{\textit{A}} + \textbf{\textit{B}}$

        Множество, состоящее из всех тех и только тех элементов, которые принадлежат как множеству \textit{\textbf{A}}, так и множеству \textit{\textbf{B}}, называется пересечением множеств \textit{\textbf{A}} и \textit{\textbf{B}} и обозначается $\textbf{\textit{A}} \cap \textbf{\textit{B}}$ или $\textbf{\textit{A}}\textbf{\textit{B}}$. Если $\textbf{\textit{A}} \cap \textbf{\textit{B}} = \O$, то говорят, что множества \textit{\textbf{A}} и \textit{\textbf{B}} не пересекаются.

        Множество, состоящее из всех элементов множества \textit{\textbf{A}}, не принадлежащих множеству \textit{\textbf{B}}, называется разностью множеств \textit{\textbf{A}} и \textit{\textbf{B}} и обозначается $\textbf{\textit{A}} \setminus \textbf{\textit{B}}$.

        Если $\textit{\textbf{A}} \subset \textit{\textbf{B}}$, то разность $\textbf{\textit{B}} \setminus \textbf{\textit{A}}$ называют дополнением множества \textit{\textbf{A}} до множества \textit{\textbf{B}} и обозначают $\textbf{\textit{A}}_B^\prime$.
        В тех случаях, когда рассматриваются только подмножества некоторого основного множества \textit{\textbf{U}}, дополнение множества \textit{\textbf{M}} до множества \textit{\textbf{U}} называют просто дополнением \textit{\textbf{M}} и вместо $\textbf{\textit{M}}_U^\prime$ пишут просто $\textbf{\textit{A}}^\prime$,

        Непосредственно из определения дополнения множества следуют равенства (\ref{eqs_1}):
        \begin{equation}
            \label{eqs_1}
            \textbf{\textit{M}} \cup \textbf{\textit{M}}^\prime = \textbf{\textit{U}}, \textbf{\textit{M}} \cap \textbf{\textit{M}}^\prime = \O, (\textbf{\textit{M}}^\prime)^\prime = \textbf{\textit{M}},
        \end{equation}

    \end{frame}

    \subsection{Эквивалентные и неэквивалентные множества}
    \begin{frame}{Эквивалентные и неэквивалентные множества}
        \fontsize{10}{8}\selectfont Множество, состоящее из всех тех и только тех элементов, которые принадлежат хотя бы одному из множеств \textit{\textbf{A}} и \textit{\textbf{B}}, называется
        объединением множеств \textit{\textbf{A}} и \textit{\textbf{B}} и обозначается $\textbf{\textit{A}} \cup \textbf{\textit{B}}$ или $\textbf{\textit{A}} + \textbf{\textit{B}}$

        Множество, состоящее из всех тех и только тех элементов, которые принадлежат как множеству \textit{\textbf{A}}, так и множеству \textit{\textbf{B}}, называется пересечением множеств \textit{\textbf{A}} и \textit{\textbf{B}} и обозначается $\textbf{\textit{A}} \cap \textbf{\textit{B}}$ или $\textbf{\textit{A}}\textbf{\textit{B}}$. Если $\textbf{\textit{A}} \cap \textbf{\textit{B}} = \O$, то говорят, что множества \textit{\textbf{A}} и \textit{\textbf{B}} не пересекаются.

        Множество, состоящее из всех элементов множества \textit{\textbf{A}}, не принадлежащих множеству \textit{\textbf{B}}, называется разностью множеств \textit{\textbf{A}} и \textit{\textbf{B}} и обозначается $\textbf{\textit{A}} \setminus \textbf{\textit{B}}$.

        Если $\textit{\textbf{A}} \subset \textit{\textbf{B}}$, то разность $\textbf{\textit{B}} \setminus \textbf{\textit{A}}$ называют дополнением множества \textit{\textbf{A}} до множества \textit{\textbf{B}} и обозначают $\textbf{\textit{A}}_B^\prime$.
        В тех случаях, когда рассматриваются только подмножества некоторого основного множества \textit{\textbf{U}}, дополнение множества \textit{\textbf{M}} до множества \textit{\textbf{U}} называют просто дополнением \textit{\textbf{M}} и вместо $\textbf{\textit{M}}_U^\prime$ пишут просто $\textbf{\textit{A}}^\prime$,

        Непосредственно из определения дополнения множества следуют равенства (\ref{eqs_1}):
        \begin{equation}
            \label{eqs_1}
            \textbf{\textit{M}} \cup \textbf{\textit{M}}^\prime = \textbf{\textit{U}}, \textbf{\textit{M}} \cap \textbf{\textit{M}}^\prime = \O, (\textbf{\textit{M}}^\prime)^\prime = \textbf{\textit{M}},
        \end{equation}
    \end{frame}

    \subsection{Система множеств.}
    \begin{frame}{Система множеств.}
        \fontsize{10}{8}\selectfont Пусть дано множество \textbf{\textit{S} = \{\textbf{\textit{s}}\}}, называемое множеством индексов, и каждому индексу \textbf{\textit{s}} сопоставлено множество  \textbf{\textit{$A_s$}}. Множество \{$A_s$\},  элементами которого являются множества \textbf{\textit{$A_s$}}, $s \in S$ называют системой или семейством множеств. Понятия объединения и пересечения двух множеств обобщаются на случай произвольной конечной или бесконечной системы множеств следующим образом.

        Объединением системы множеств $A_s$, $s \in S$, называется множество всех элементов, принадлежащих хотя бы одному из множеств системы.

        Пересечением системы множеств $A_s$, $s \in S$, называется множество всех элементов, содержащихся в каждом множестве системы.

        Объединение и пересечение системы множеств $A_s$, $s \in S$, обозначают соответственно (\ref{eqs_4})

        \begin{equation}
            \label{eqs_4}
            \bigcup_{s \in S} A_s,  \bigcap_{s \in S} A_s.
        \end{equation}
    \end{frame}

    \section{Комбинаторика}

    \subsection{Упорядоченные множества}
    \begin{frame}{Упорядоченные множества}
        \fontsize{10}{8}\selectfont Множество называется упорядоченным, если для любых двух его элементов \textit{a} и \textit{b} установлено отношение порядка $\textit{a} \leq \textit{b}$ или $\textit{b} \leq \textit{a}$ \textit{a} не превосходит \textit{b} или \textit{b} не превосходит \textit{a}, обладающее свойствами:

        \begin{enumerate}
            \item рефлексивности: $\textit{a} \leq \textit{a}$, т.е. любой элемент не превосходит
            самого себя;
            \item антисимметричности: если $\textit{a} \leq \textit{b}$ и $\textit{b} \leq \textit{a}$, то элементы \textit{a} и \textit{b} равны;
            \item транзитивности: если $\textit{a} \leq \textit{b}$, $\textit{b} \leq \textit{c}$, то $\textit{a} \leq \textit{c}$.
        \end{enumerate}
    \end{frame}

    \subsection{Размещения и перестановки}
    \begin{frame}{Размещения и перестановки}
        \fontsize{10}{8}\selectfont Пусть имеется множество, содержащее \textit{n} элементов. Каждое его упорядоченное подмножество, состоящее из \textit{k} элементов, называется размещением из п элементов по \textit{k} элементов.

        Число размещений из \textit{n} элементов по \textit{k} элементов обозначается $A_n^k$ И вычисляется по формуле (\ref{eqs_6})

        \begin{equation}
            \label{eqs_6}
            A_n^k=n(n-1)(n-2)\ldots(n-(k-1))
        \end{equation}

        Размещения из \textit{n} элементов по \textit{n} элементов называются перестановками из \textit{n} элементов. Число перестановок из \textit{n} элементов обозначается $P_n$ и вычисляется по формуле (\ref{eqs_7})

        \begin{equation}
            \label{eqs_7}
            P_n=1\cdot2\cdot\ldots\cdot n= n!.
        \end{equation}
    \end{frame}

    \subsection{Сочетания}
    \begin{frame}{Сочетания}
        \fontsize{10}{8}\selectfont Пусть имеется множество, состоящее из \textit{n} элементов. Каждое его подмножество, содержащее \textit{k} элементов, называется сочетанием из \textit{n}  элементов по \textit{k} элементов.

        Число всех сочетаний из \textit{n} элементов по \textit{k} элементов обозначается символом $C_n^k$ и вычисляется по формуле (\ref{eqs_8})

        \begin{equation}
            \label{eqs_8}
            C_n^k=\frac{n!}{k!(n-k)!},
        \end{equation}
        или по формуле (\ref{eqs_9})
        \begin{equation}
            \label{eqs_9}
            C_n^k=\frac{n(n-1)(n-2)\ldots(n-k+1)}{k!}.
        \end{equation}

        Справедливы равенства (\ref{eqs_10}):
        \begin{equation}
            \label{eqs_10}
            C_n^k=C_n^{k-1}, C_{n+1}^{k+1}=C_n^{k+1}+C_n^k, k < n.
        \end{equation}
    \end{frame}

\end{document}
